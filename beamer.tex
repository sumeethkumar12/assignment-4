\documentclass{beamer}

% Theme choice:
\usetheme{CambridgeUS}

% Title page details: 
\title{Assignment 4 (AI1110)} 
\author{sumeeth kumar\\
        ai21btech11008}
\date{\today}
\logo{\large \LaTeX{}}


\begin{document}

% Title page frame
\begin{frame}
    \titlepage 
\end{frame}

% Remove logo from the next slides
\logo{}


% Outline frame
\begin{frame}{Outline}
    \tableofcontents
\end{frame}


% Lists frame
\section{Question}
\begin{frame}{Question(example:1.4):}

\begin{itemize}
    \item Given n particles and $m > n$ boxes. we place at random each particle in one of the 
boxes. We wish to find the probability p that in n pre selected boxes, one and only one particle will be found. 
\end{itemize}
\end{frame}
\section{Solution}
\begin{frame}{Solution:}
   \begin{enumerate}
       \item If we accept as outcomes all possible ways of placing n particles in m boxes distinguishing the identity of each particle, then
       \begin{align}
           p =\dfrac{n!}{m^n}        
           \end{align}
        \item If we assume that the particles are not distinguishable, that is, if all their permutations count as one, then
        \begin{align}
            p =\dfrac{(m-1)!(n)!}{(m+n-1)!} 
        \end{align}
         \item If we do not distinguish between the particles and also we assume that in each box we are allowed to place at most one particle, then 
         \begin{align}
             p =\dfrac{(n)!(m-n)!}{m!}
         \end{align}
     \end{enumerate}
    
\end{frame}
\end{document}
